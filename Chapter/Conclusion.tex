\chapter{Fazit}
Im letzten Kapitel dieser Arbeit wird anhand der Ergebnisse bezug zur Lösung der Problemstellung genommen. Ebenso wird die in Kapitel \ref{sec:goal} aufgestellte Zielstellung überprüft, so dass anhand dieser der Erfolg der Arbeit gemessen werden kann.

\section{Schlussfolgerung}
Im Rahmen dieser Arbeit wurden dem Leser in Kapitel \ref{chap:basics} relevante technische Grundlagen vermittelt, um den Kontextm, die Herangehnsweise sowie angewandte Technologien besser zu verstehen. Um den Kernpunkt des Themas bearbeiten zu können, war eine Konzeption sowie Implementaion eines automatisierten Testerstellungssystems notwendig. Diese wurde detailreich und nachvollziehbar in den Kapiteln \ref{chapter:concept} und \ref{chapter:impl} durchgeführt und ermöglichte die Grundlage einer anschließenden Analyse. Die Durchführung basierte anhand verschiedener Projekte mit unterschiedlichen Komplexitätsgraden, durch welche eine aussagekräftige Anzahl an Tests generiert werden konnte. Zur Evaluierung der Ergebnisse wurde ein fest definiertes Bewertungsverfahren mit klar gesetzten Metriken genutzt sowie im Anschluss daran der Vergleich zur reellen Arbeitswelt gezogen. Betrachtet man die am Anfang der Arbeit beschriebenen Ziele mit den eben genannten Umsetzungen kann schlussfolgernd gesagt werden, dass sowohl die Grundlage einer aussagekräftigen Analyse als auch die Analyse selbst erfolgreich umgesetzt wurde.\\\\
Die im Kapitel \ref{sec:prob} beschriebene Problemstellung eröffnete die Frage inwiefern Sprachmodelle die Arbeit von Testern, im Kontext der Generierung von \textit{JUnit Tests} ersetzen können und in welcher Qualität sie deren Arbeit übernehmen.
\section{Ausblick}