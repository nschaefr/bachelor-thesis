\chapter{Implementation}\label{chapter:impl}
Die Implementation dient der Umsetzung des im Kapitel \ref{chapter:concept} erstellten Konzepts, sodass ein Testerstellungssystem \textit{Unitcraft} zur Analyse des Hauptthemas bereitgestellt wird. Es erfolgt abschließend ein Testlauf des Programms, um die Funktionalität gewährleisten zu können.

\section{Funktionalitäten}
Die im Konzept festgelegten Zielfunktionalitäten müssen mit Blick auf den Programmablaufplan [Abb. \ref{fig:pap}] zur erfolgreichen Bearbeitung der Arbeit umgesetzt werden. Dabei ist zu beachten, dass alle Funktionalitäten abgedeckt sind sowie alle Komponenten dem Konzept entsprechend zusammenarbeiten.