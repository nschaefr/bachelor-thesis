\chapter{Konzeption}
Der Inhalt von Kapitel 3 stellt die Konzeption eines Testerstellungssystems namens \textit{Unitcraft} zur Bearbeitung des Hauptthemas dar. Dies geschieht in einer nachvollziehbaren Weise, so dass das erstellte Konzept reproduzierbar ist.

\section{Voraussetzungen}
Im Mittelpunkt der Arbeit befindet sich die Programmiersprache Java. Zur Vereinfachung der Analyse und Ausführung der generierten Tests werden die Java-Projekte auf das Build-Automation-Tool \textit{Apache Maven} beschränkt. Hierbei wird die klassische Maven Projektstruktur eingehalten. Diese besteht zunächst aus dem \textit{src}-Verzeichnis, welches sich in das \textit{main}-Verzeichnis, in dem sich der Java-Programmcode bzw. die dazugehörigen Ressourcen befinden, und in das \textit{test}-Verzeichnis unterteilt. \cite{MavenIntroductionStandard} [Abb. \ref{fig:dir}]\\ Somit wird eine klare Trennung zwischung Programmcode und Testcode gewährleistet und deutlich, in welchem Verzeichnis die Tests abgelegt werden müssen. \begin{figure}[h]
    \centering
    \begin{minipage}{4cm}
    \dirtree{%
    .1 src.
    .2 main.
    .3 java.
    .3 resources.
    .2 test.
    .3 java.
    .3 resources.
    }
    \end{minipage}
    \caption{\textit{Maven} Projektstruktur}
    \label{fig:dir}
\end{figure} \newpage
Um ein umfangreiches Generieren von Tests zu ermöglichen, ist die Einbindung von Frameworks als \textit{Dependencies} (engl. Abhängigkeiten) in der pom.xml essenziell. Neben dem Nutzen von JUnit-Jupiter wird ein Einbinden von Mockito vorausgesetzt, so dass die Verwendung von \textit{mocks} (engl. Attrappen) ermöglicht wird.\\
Zur Generierung von Metriken wird SonarQube in das Konzept eingebaut. Dabei wird eine weiteres \textit{Plugin} benötigt, um einen Coverage-Report erzeugen zu können. Die \textit{Java Code Coverage Library} (kurz JaCoCo) ermöglicht das Erstellen eines Code-Reports und stellt SonarQube alle notwendigen Daten bereit.
Die Implementierung von Plugins und Dependencies erfolgt im Kapitel \ref{chapter:impl}.

\section{Anforderungsanalyse}
Um das Testerstellungssystem praktikabel anwendbar zu gestalten, muss zunächst die Zielgruppe der Arbeit betrachtet werden. Es handelt sich dabei um Personen, die fachlich zugeordnet sind, und somit eine Wissensbasis im Bereich der Informatik besitzen. Aufgrunddessen wird auf eine nutzerfreundliche Oberfläche verzichtet und somit der Fokus auf die Programmlogik gelenkt.\\ Ein Testerstellungssystem als Kommandozeilen-Tool bietet hier eine vereinfachte Anwendung innerhalb des Projekts und eröffnet die Möglichkeit einer zukünftigen Einbindung in Automationsprozesse wie bspw. der \textit{Continous Integration} (engl. kurz CI).\\ Die Wahl der Programmiersprache fällt hierbei auf Python. Python ist eine \textit{high-level}, interpretierte und dynamische Programmiersprache, welche Vorteile wie bspw. zahlreiche \textit{Libraries} (engl. Bibliotheken) für LLM-Schnittstellen, eine große aktive Community sowie eine einfache Lesbarkeit mit sich bringt. Die Umsetzung eines Python-Kommandozeilen-Tools erfordert eine detaillierte Anforderungsanalyse, um die Zielfunktionalitäten zu gewährleisten. \cite*{PythonLanguageAdvantages2017}\\\\
Um die zu verwendende Prompt-Technik und Temperature festlegen zu können, benötigen zu Beginn eine \textbf{Nutzerabfrage zum Initialisieren der Prompt- und Temperaturevariablen}. Dabei kann zwischen den vorher definierten Promptdesigns gewählt werden.