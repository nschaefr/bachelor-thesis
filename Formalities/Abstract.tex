%!TeX root = ./../MusterAbschlussarbeit.tex

%##########################################################
% Inhalt
%##########################################################

\clearpage
\chapter*{Kurzfassung}
Die Bachelorarbeit untersucht die Qualität von Sprachmodellen im Kontext der Generierung von \textit{Unit Tests} für \textit{Java} Anwendungen. Ziel der Arbeit ist es, zu analysieren, inwieweit \textit{JUnit Tests} durch den Einsatz von \textit{Large Language Models} automatisiert generiert werden können und daraus abzuleiten mit welcher Qualität sie die Arbeit von Software-Testern übernehmen und ersetzen. Hierzu wird ein automatisiertes Testerstellungssystem in Form eines \textit{Python}-Kommandozeilen-\textit{Tools} konzipiert sowie implementiert, welches mithilfe von API-Anfragen Testfälle generiert. Um die Qualität des Sprachmodells messen zu können, werden die generierten Tests ohne manuellen Einfluss übernommen. Als Grundlage der Evaluierung findet eine Durchführung statt, in der für 3 \textit{Java-Maven} Projekte mit unterschiedlichen Komplexitätsgraden Tests generiert werden. Die anschließende Analyse besteht aus einem festen Bewertungsverfahren, welches die Testcodeabdeckung sowie Erfolgsquote evaluiert und mit manuellen Tests vergleicht. Die Ergebnisse zeigen, dass \textit{Large Language Models} in der Lage sind, \textit{JUnit Tests} mit einer zufriedenstellenden Testabdeckung zu generieren, jedoch eine unzureichende Erfolsquote im Vergleich zu manuellen Tests aufweisen. Es wird deutlich, dass Sprachmodelle aufgrund von Qualitätsmängeln bei der Generierung von Testcode die Arbeit von Software-Testern nicht vollständig ersetzen können. Jedoch bieten sie eine Möglichkeit Testerstellungsprozesse zu übernehmen, welche mit einer anschließenden manuelle Nachkontrolle enden und reduzieren somit den Arbeitsaufwand der Tester. Die Arbeit schließt mit einem kurzen Ausblick über die Potenziale des Einsatzes von Sprachmodellen in der Testerstellung.\\\\
\keywords{\textit{Large Language Models}, \textit{JUnit}, \textit{Software Testing}, \textit{Code Coverage}, \textit{OpenAI}, automatisierte Testgenerierung, \textit{Prompt Engineering}}
